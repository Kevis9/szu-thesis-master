% !Mode:: "TeX:UTF-8"

%%%% 此处是论文封面格式的定义,数据填写的部分在下方 %%%

\phantomsection
\makeatletter

\def\thesisTitle#1{\def\@thesisTitle{#1}}\def\@thesisTitle{}
\def\ctitle#1{\def\@ctitle{#1}}\def\@ctitle{}
\def\cdegree#1{\def\@cdegree{#1}}\def\@cdegree{}
\def\majorName#1{\def\@majorName{#1}}\def\@majorName{}
\def\authorName#1{\def\@authorName{#1}}\def\@authorName{}
\def\supervisor#1{\def\@supervisor{#1}}\def\@supervisor{}

\def\cheading#1{\def\@cheading{#1}}\def\@cheading{}
\def\security#1{\def\@security{#1}}\def\@security{}
\def\schoolcode#1{\def\@schoolcode{#1}}\def\@schoolcode{}
\def\college#1{\def\@college{#1}}\def\@college{}
\def\UDCnumber#1{\def\@UDCnumber{#1}}\def\@UDCnumber{}
\def\classnumber#1{\def\@classnumber{#1}}\def\@classnumber{}

% 定义makecover以生成封面
\def\makecover{
	
	\clearpage
	
	\pdfbookmark[-1]{封面}{ctitle} % pdf文件的书签
	
	\begin{titlepage}
		\begin{center}
			
			\setlength{\@title@width}{3.5cm}
			{
				\begin{tabular}{lcclc}
					\xiaosi\hei{分类号}&  \underline{\makebox[\@title@width][c]{\@classnumber}}&\qquad \qquad \qquad \qquad \qquad & 
					\xiaosi\hei{学校代码}&  \underline{\makebox[\@title@width][c]{\@schoolcode}}\\
					\xiaosi\hei{U~~D~~C}& \underline{\makebox[\@title@width][c]{\@UDCnumber}}&\qquad \qquad \qquad \qquad \qquad & 
					\xiaosi\hei{密~~~~~~级}&  \underline{\makebox[\@title@width][c]{\@security}} \\
				\end{tabular}
			}
			
			\vspace*{3cm}
			{\song\erhao \@cheading}
			
			\vspace*{1cm}
			
			\begin{center}
				\begin{spacing}{1.5}
					\hei\sanhao \@thesisTitle%标题可以使用断字
				\end{spacing}
			\end{center}
			
			\begin{center}
				\begin{spacing}{1.5}
					\song\sanhao \@authorName%标题可以使用断字
				\end{spacing}
			\end{center}
			
			%\makebox[宽度][位置]{文本}中可指定盒子宽度,文本在盒子中的位置(l:左端;r:右端;s:两端,默认是居中)
			\vspace{\baselineskip}
			\setlength{\@title@width}{6.5cm}
			{
				\begin{spacing}{3}
					\makebox[3.3cm][s]{\sanhao\song{学~~~~位~~~~类~~~~别}} \sanhao\song\underline{\makebox[\@title@width]{\@cdegree}} \\
					\makebox[3.3cm][s]{\sanhao\song{专~~~~业~~~~名~~~~称}} \sanhao\song\underline{\makebox[\@title@width]{\@majorName}} \\
					\makebox[3.3cm][s]{\sanhao\song{学院(系、所)}} \sanhao\song\underline{\makebox[\@title@width]{\@college}} \\
					\makebox[3.3cm][s]{\sanhao\song{指导老师}} \sanhao\song\underline{\makebox[\@title@width]{\@supervisor}} \\
				\end{spacing}
			}
		\end{center}
		
		\clearpage
		
	\end{titlepage}	
}
\makeatother


%%% 此处是数据填写部分 

\classnumber{TP37}
\schoolcode{10590}
\UDCnumber{UDCnumber}
\security{公开}

\cheading{深圳大学硕士学位论文}
\thesisTitle{论文名称} %封面用论文标题,自己可手动断行
\ctitle{中文页眉标题}  %页眉标题无需断行
\cdegree{学位类别}
\college{计算机科学与软件学院} %学院名称
\majorName{计算机科学与技术}   %专业
\authorName{张~~三}   %学生姓名
\supervisor{李四教授} %导师姓名

% 使用makecover生成封面
\makecover