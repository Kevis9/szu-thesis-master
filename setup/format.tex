% !Mode:: "TeX:UTF-8"
%  Authors: 杜家宜   Jiayi Du: Max_dujiayi@gmail.com     湖南大学2010级计算机科学与技术专业博士生

% Modified by Jon, 2021/01/03


%%%%%%%%%% 字体命令定义 %%%%%%%%%%%%%%%%%
\newcommand{\song}{\CJKfamily{song}}    % 宋体
\newcommand{\fs}{\CJKfamily{fs}}        % 仿宋体
\newcommand{\kai}{\CJKfamily{kai}}      % 楷体
\newcommand{\hei}{\CJKfamily{hei}}      % 黑体
\newcommand{\li}{\CJKfamily{li}}        % 隶书
\newcommand{\yihao}{\fontsize{26pt}{26pt}\selectfont}       % 一号, 1.倍行距
\newcommand{\xiaoyi}{\fontsize{24pt}{24pt}\selectfont}      % 小一, 1.倍行距
\newcommand{\erhao}{\fontsize{22pt}{22pt}\selectfont}       % 二号, 1.倍行距
\newcommand{\xiaoer}{\fontsize{18pt}{18pt}\selectfont}      % 小二, 单倍行距
\newcommand{\sanhao}{\fontsize{16pt}{16pt}\selectfont}      % 三号, 1.倍行距
\newcommand{\xiaosan}{\fontsize{15pt}{15pt}\selectfont}     % 小三, 1.倍行距
\newcommand{\sihao}{\fontsize{14pt}{14pt}\selectfont}       % 四号, 1.0倍行距
\newcommand{\xiaosi}{\fontsize{12pt}{12pt}\selectfont}  	% 小四, 1.倍行距
\newcommand{\wuhao}{\fontsize{10.5pt}{10.5pt}\selectfont}   % 五号, 单倍行距
\newcommand{\xiaowu}{\fontsize{9pt}{9pt}\selectfont}        % 小五, 单倍行距
\setlength{\headheight}{20pt}
%\CJKcaption{gb_452}
\CJKtilde  % 重新定义了波浪符~的意义
\newcommand\prechaptername{第}
\newcommand\postchaptername{章}

% 调整罗列环境的布局 % Modified by Li Jianmin
\setitemize{leftmargin=0em,itemindent=3em,itemsep=0em,partopsep=0em,parsep=0em,topsep=-0em}
\setenumerate{leftmargin=0em,itemindent=3em,itemsep=0em,partopsep=0em,parsep=0em,topsep=0em}


%避免宏包 hyperref 和 arydshln 不兼容带来的目录链接失效的问题。
\def\temp{\relax}
\let\temp\addcontentsline
\gdef\addcontentsline{\phantomsection\temp}


% 自定义项目列表标签及格式 \begin{publist} 列表项 \end{publist}
\newcounter{pubctr} % 自定义新计数器
\newenvironment{publist}{%%%%% 定义新环境
\begin{list}{[\arabic{pubctr}]} %% 标签格式
    {
     \usecounter{pubctr}
     \setlength{\leftmargin}{2em}     % 左边界 \leftmargin =\itemindent + \labelwidth + \labelsep
     \setlength{\itemindent}{0em}     % 标号缩进量
     \setlength{\labelsep}{1em}       % 标号和列表项之间的距离,默认0.5em
     \setlength{\rightmargin}{0em}    % 右边界
     \setlength{\topsep}{0ex}         % 列表到上下文的垂直距离
     \setlength{\parsep}{0ex}         % 段落间距
     \setlength{\itemsep}{0ex}        % 标签间距
     \setlength{\listparindent}{0pt} % 段落缩进量
    }}
{\end{list}}%%%%%


\makeatletter
\renewcommand\normalsize{
  \@setfontsize\normalsize{12pt}{12pt} 	% 此处用于设置正文字体,小四号字体
  \setlength\abovedisplayskip{4pt}
  \setlength\abovedisplayshortskip{4pt}
  \setlength\belowdisplayskip{\abovedisplayskip}
  \setlength\belowdisplayshortskip{\abovedisplayshortskip}
  \setlength{\baselineskip}{23pt}		% 设置行距和段落间垂直距离
    \let\@listi\@listI}
\def\defaultfont{\renewcommand{\baselinestretch}{1.65}\normalsize\selectfont}
\makeatother

%%%%%%%%%%%%% 目录格式设置 %%%%%%%%%%%%%%%%%
\renewcommand{\contentsname}{目\quad录}
\setcounter{tocdepth}{2}
\titlecontents{chapter}[0em]{\sihao\song}%
             {\prechaptername~~\thecontentslabel~~\postchaptername~~~}{} %
             {\titlerule*[5pt]{$\cdot$}\sihao\contentspage}
\titlecontents{section}[2.0em]{\sihao\song} %
            {\thecontentslabel\quad}{} %
            {\hspace{.25em}\titlerule*[5pt]{$\cdot$}\sihao\contentspage}
\titlecontents{subsection}[3.25em]{\sihao\song} %
            {\thecontentslabel\quad}{} %
            {\hspace{.25em}\titlerule*[5pt]{$\cdot$}\sihao\contentspage}
\renewcommand{\cftdotsep}{1.1}
\renewcommand{\listfigurename}{插图索引}
\setcounter{lofdepth}{1}
\renewcommand{\listtablename}{附表索引}


%%删除表格和插图因章不同中的空行%%%
\makeatletter
\def\@chapter[#1]#2
{\ifnum \c@secnumdepth >\m@ne
	\if@mainmatter
	\refstepcounter{chapter}%
	\typeout{\@chapapp\space\thechapter.}%
	\addcontentsline{toc}{chapter}%
	{\protect\numberline{\thechapter}#1}%
	\else
	\addcontentsline{toc}{chapter}{#1}%
	\fi
	\else
	\addcontentsline{toc}{chapter}{#1}%
	\fi
	\chaptermark{#1}%
	\if@twocolumn
	\@topnewpage[\@makechapterhead{#2}]%
	\else
	\@makechapterhead{#2}%
	\@afterheading
	\fi}
\makeatother


%%%%%%%%%% 章节标题格式设置 %%%%%%%%%%%%%%%%%
\setcounter{secnumdepth}{4}
\setlength{\parindent}{2em}
\renewcommand{\chaptername}{\prechaptername\arabic{chapter}\postchaptername}
% modified by Jon,对章节标题的字体进行调整
\titleformat{\chapter}{\centering\xiaosan\hei}{\chaptername}{1em}{}		% 一级标题(章):小三号黑体
\titlespacing{\chapter}{0pt}{0pt}{18pt}
\titleformat{\section}{\sihao\song\bfseries}{\thesection}{1em}{}		% 二级标题(节):四号宋体(加粗),\bfseries表示加粗
\titlespacing{\section}{0pt}{12pt}{12pt}
\titleformat{\subsection}{\xiaosi\song\bfseries}{\thesubsection}{0.5em}{}   % 三级标题(节内小节):小四号宋体(加粗) 
\titlespacing{\subsection}{0pt}{6pt}{6pt}
\titleformat{\subsubsection}{\xiaosi\song}{\thesubsubsection}{0.5em}{}
\titlespacing{\subsubsection}{0pt}{6pt}{6pt}



%%%%%%%%%% 表格,图表,公式格式设置 %%%%%%%%%%%%%%%%%
\renewcommand{\tablename}{表} % 插表题头
\renewcommand{\figurename}{图} % 插图题头
\renewcommand{\thefigure}{\arabic{chapter}.\arabic{figure}} % 使图编号为 7.1 的格式 %\protect{~}
\renewcommand{\thetable}{\arabic{chapter}.\arabic{table}}%使表编号为 7.1 的格式
\renewcommand{\theequation}{\arabic{chapter}.\arabic{equation}}%使公式编号为 7.1 的格式
\renewcommand{\thesubfigure}{\alph{subfigure})}%使子图编号为a)的格式
\renewcommand{\thesubtable}{(\alph{subtable})} %使子表编号为a)的格式
\makeatletter
\renewcommand{\p@subfigure}{\thefigure~} %使子图引用为 7.1 a) 的格式,母图编号和子图编号之间用~ 加一个空格
\makeatother



%% 定制浮动图形和表格标题样式
% \fontsize{10.5pt}{10.5pt}设置字体大小,\linespread{1.2}设置行间距
\makeatletter
\long\def\@makecaption#1#2{%
   \vskip\abovecaptionskip
   \sbox\@tempboxa{\centering\fontsize{10.5pt}{10.5pt}\linespread{1.2}\selectfont{}\song{#1~~#2} }	% 图表标题字体格式
   \ifdim \wd\@tempboxa >\hsize
     \centering\fontsize{10.5pt}{10.5pt}\linespread{1.2}\selectfont{}\song{#1~~#2} \par
   \else
     \global \@minipagefalse
     \hb@xt@\hsize{\hfil\box\@tempboxa\hfil}%
   \fi
   \vskip\belowcaptionskip}
\makeatother
\captiondelim{~~~~} %用来控制longtable表头分隔符

\BeforeBeginEnvironment{tabular}{\wuhao}	% 表格内文字字体格式
\BeforeBeginEnvironment{table}{
	\setlength{\abovecaptionskip}{0pt}	% 表格标题上方的空白
	\setlength{\belowcaptionskip}{5pt} % 标题下方的空白
}

\BeforeBeginEnvironment{figure}{
	\setlength{\abovecaptionskip}{0pt}	% 图片标题上方的空白
	\setlength{\belowcaptionskip}{0pt} % 标题下方的空白
}

%%%%%%%%%% Theorem Environment %%%%%%%%%%%%%%%%%
\theoremstyle{plain}
\theorembodyfont{\xiaosi\song}%\rmfamily}
\theoremheaderfont{\xiaosi\hei}%\rmfamily}
\setlength{\theorempreskipamount}{0em} %调整定理环境与上文的距离
\setlength{\theorempostskipamount}{0em} %调整定理环境与下文的距离
\newtheorem{theorem}{定理~}[chapter]
\newtheorem{lemma}{引理~}[chapter]
\newtheorem{axiom}{公理~}[chapter]
\newtheorem{proposition}{命题~}[chapter]
\newtheorem{corollary}{推论~}[chapter]
\newtheorem{definition}{\hskip 2em 定义~}[chapter]
\newtheorem{conjecture}{猜想~}[chapter]
\newtheorem{example}{例~}[chapter]
\newtheorem{remark}{注~}[chapter]
\floatname{algorithm}{算法}% 将英文的algorithm改为算法
\renewcommand{\algorithmicrequire}{\textbf{Input:}}
\renewcommand{\algorithmicensure}{\textbf{Output:}}
\newcommand{\tabincell}[2]{\begin{tabular}{@{}#1@{}}#2\end{tabular}}%表格合并
\newenvironment{proof}{\noindent{\hei 证明:}}{\hfill $ \square $ \vskip 4mm}
\theoremsymbol{$\square$}


\algrenewcommand\alglinenumber[1]{\fontsize{10.5pt}{15pt}\selectfont #1:}	% 算法行号字体设置
\algrenewcommand{\ALG@beginalgorithmic}{\fontsize{10.5pt}{15pt}\selectfont \song}	% 算法内容字体设置

%%%%%%%%%% Page: number, header and footer  页码%%%%%%%%%%%%%%%%%

%\frontmatter 或 \pagenumbering{roman}
%\mainmatter 或 \pagenumbering{arabic}
\makeatletter
\renewcommand\frontmatter{\clearpage
  \@mainmatterfalse
  \pagenumbering{Roman}} % 正文前罗马字体编号
\makeatother



% 参考文献样式
%%%%%%%%%% References %%%%%%%%%%%%%%%%%
\renewcommand{\bibname}{参\quad 考\quad 文\quad 献}
\makeatletter
\renewenvironment{thebibliography}[1]{%
   \chapter*{\centering\xiaosan\hei\bfseries \bibname}  % modified by Jon, 标题小三黑体加粗
   \wuhao		% modified by Jon, 参考文献引用使用5号字体
   \list{\@biblabel{\@arabic\c@enumiv}}%
        {\renewcommand{\makelabel}[1]{##1\hfill}
         \setlength{\baselineskip}{21pt}
         \settowidth\labelwidth{0.5cm}
         \setlength{\labelsep}{0pt}
         \setlength{\itemindent}{0pt}
         \setlength{\leftmargin}{\labelwidth+\labelsep}
         \addtolength{\itemsep}{-0.7em}
         \usecounter{enumiv}%
         \let\p@enumiv\@empty
         \renewcommand\theenumiv{\@arabic\c@enumiv}}%
    \sloppy\frenchspacing
    \clubpenalty4000%
    \@clubpenalty \clubpenalty
    \widowpenalty4000%
    \interlinepenalty4000%
    \sfcode`\.\@m}
   {\def\@noitemerr
     {\@latex@warning{Empty `thebibliography' environment}}%
    \endlist\frenchspacing}
\makeatother

\addtolength{\bibsep}{5pt} % 增加参考文献间的垂直间距
\setlength{\bibhang}{2em} %每个条目自第二行起缩进的距离


% 参考文献引用作为上标出现
\newcommand{\mycite}[1]{\scalebox{1.3}[1.3]{\raisebox{-0.65ex}{\cite{#1}}}}
%\newcommand{\citenormal}[1]{\cite{#1}}
%\makeatletter
%   \def\@cite#1#2{\textsuperscript{[{#1\if@tempswa , #2\fi}]}}
%\makeatother


%% 引用格式
\bibpunct{[}{]}{,}{s}{}{,}


\makeatletter
\newlength{\@title@width}
\def\@put@covertitle#1{\makebox[\@title@width][s]{#1}}

%%%%%%%%%%%%%%%%%%% 设置页眉的样式 %%%%%%%%%%%%%%%%%%%

\def\headrule{{\if@fancyplain\let\headrulewidth\plainheadrulewidth\fi%
		\hrule\@height 0.5pt \@width\headwidth\vskip1pt %线条粗0.5pt
	}
	\vspace{0mm} %双线与下面正文之间的垂直间距
}

\fancypagestyle{plain}{% 设置页页眉页脚的字体
	\fancyhf{}%
	\fancyhead[CO]{\song\wuhao \@cnpageheader}		% 奇数页的页眉标题也使用@cnpageheader
	\fancyhead[CE]{\song\wuhao \@cnpageheader}
	\fancyfoot[C]{\song\wuhao ~\thepage~} 		%首页页脚格式
}

\clearpage
\makeatother


%%% 此处是论文封面格式的定义,数据部分在cover.tex %%%

\phantomsection
\makeatletter

\def\thesisTitle#1{\def\@thesisTitle{#1}}\def\@thesisTitle{}
\def\cdegree#1{\def\@cdegree{#1}}\def\@cdegree{}
\def\majorName#1{\def\@majorName{#1}}\def\@majorName{}
\def\authorName#1{\def\@authorName{#1}}\def\@authorName{}
\def\supervisor#1{\def\@supervisor{#1}}\def\@supervisor{}

\def\cheading#1{\def\@cheading{#1}}\def\@cheading{}
\def\security#1{\def\@security{#1}}\def\@security{}
\def\schoolcode#1{\def\@schoolcode{#1}}\def\@schoolcode{}
\def\college#1{\def\@college{#1}}\def\@college{}
\def\UDCnumber#1{\def\@UDCnumber{#1}}\def\@UDCnumber{}
\def\classnumber#1{\def\@classnumber{#1}}\def\@classnumber{}

% 定义makecover以生成封面
\def\makecover{
	
	\clearpage
	\pdfbookmark[-1]{封面}{thesisTitle} % pdf文件的书签
	
	\begin{titlepage}
		\begin{center}
			
			\setlength{\@title@width}{3.5cm}
			{
				\begin{tabular}{lcclc}
					\xiaosi\hei{分类号}&  \underline{\makebox[\@title@width][c]{\@classnumber}}&\qquad \qquad \qquad \qquad \qquad & 
					\xiaosi\hei{学校代码}&  \underline{\makebox[\@title@width][c]{\@schoolcode}}\\
					\xiaosi\hei{U~~D~~C}& \underline{\makebox[\@title@width][c]{\@UDCnumber}}&\qquad \qquad \qquad \qquad \qquad & 
					\xiaosi\hei{密~~~~~~级}&  \underline{\makebox[\@title@width][c]{\@security}} \\
				\end{tabular}
			}
			
			\vspace*{3cm}
			{\song\erhao \@cheading}
			
			\vspace*{1cm}
			
			\begin{center}
				\begin{spacing}{1.5}
					\hei\sanhao \@thesisTitle%标题可以使用断字
				\end{spacing}
			\end{center}
			
			\begin{center}
				\begin{spacing}{1.5}
					\song\sanhao \@authorName%标题可以使用断字
				\end{spacing}
			\end{center}
			
			%\makebox[宽度][位置]{文本}中可指定盒子宽度,文本在盒子中的位置(l:左端;r:右端;s:两端,默认是居中)
			\vspace{\baselineskip}
			\setlength{\@title@width}{6.5cm}
			{
				\begin{spacing}{3}
					\makebox[3.3cm][s]{\sanhao\song{学~~~~位~~~~类~~~~别}} \sanhao\song\underline{\makebox[\@title@width]{\@cdegree}} \\
					\makebox[3.3cm][s]{\sanhao\song{专~~~~业~~~~名~~~~称}} \sanhao\song\underline{\makebox[\@title@width]{\@majorName}} \\
					\makebox[3.3cm][s]{\sanhao\song{学院(系、所)}} \sanhao\song\underline{\makebox[\@title@width]{\@college}} \\
					\makebox[3.3cm][s]{\sanhao\song{指导老师}} \sanhao\song\underline{\makebox[\@title@width]{\@supervisor}} \\
				\end{spacing}
			}
		\end{center}
		
	\end{titlepage}	
	
	\clearpage
}
\makeatother
%%%%%%%%%%%%%%%%%%%%%%

%%%% 此处是原创性声明的格式定义,数据部分在copyright.tex %%%

\makeatletter

\def\declaretitle#1{\def\@declaretitle{#1}}\def\@declaretitle{}
\def\declarecontent#1{\def\@declarecontent{#1}}\def\@declarecontent{}
\def\authorizationtitle#1{\def\@authorizationtitle{#1}}\def\@authorizationtitle{}
\def\authorizationcontent#1{\def\@authorizationcontent{#1}}\def\@authorizationconent{}
\def\authorizationadd#1{\def\@authorizationadd{#1}}\def\@authorizationadd{}
\def\authorsigncap#1{\def\@authorsigncap{#1}}\def\@authorsigncap{}
\def\supervisorsigncap#1{\def\@supervisorsigncap{#1}}\def\@supervisorsigncap{}
\def\signdatecap#1{\def\@signdatecap{#1}}\def\@signdatecap{}

% 定义makecopyright以生成原创性说明
\def\makecopyright{
	
	\clearpage
	
	
	% 下面为原创性声明页面的格式设置,如果引入pdf文件作为原创性声明,则将此部分代码注释掉
	\pagestyle{fancy}
	\fancyhf{}
	\fancyfoot[C]{\song\xiaowu ~\thepage~}
	\renewcommand{\headrulewidth}{0pt}
	\chapter*{\song\sanhao 深圳大学学位论文原创性声明和使用授权说明}{		% modified by Jon, 原创性说明无需加入目录,使用chapter*
		\thispagestyle{empty} 		% modified by Jon: 原创性说明不需要页码
		\qquad\\
		\begin{center}\hei\xiaoer{\@declaretitle}\end{center}\par
		\song\defaultfont{\@declarecontent}\par
		\vspace*{1cm}
		{\song\xiaosi
			\@authorsigncap \makebox[2.5cm][s]{}
			\@signdatecap \makebox[2cm][s]{} 年 \makebox[1cm][s]{} 月 \makebox[1cm][s]{} 日
		}
		\vspace{0.6\baselineskip}
		\begin{center}\hei\xiaoer{\@authorizationtitle}\end{center}\par
		{
			\vspace{1.2\baselineskip}
			\song\defaultfont{\@authorizationcontent}
			\par \song\defaultfont{\@authorizationadd}
		}
		\vspace{2\baselineskip}
		
		{
			\song\xiaosi
			\@authorsigncap \makebox[3.5cm][s]{}  \@signdatecap \makebox[1.5cm][s]{} 年 \makebox[1cm][s]{} 月 \makebox[1cm][s]{} 日 \\
			\indent
			\@supervisorsigncap \makebox[3.5cm][s]{}  \@signdatecap \makebox[1.5cm][s]{} 年 \makebox[1cm][s]{} 月 \makebox[1cm][s]{} 日
		}
	}
	
	\clearpage
}

\makeatother
%%%%%%%%%%%%%%%%%%%%%%


%%% 此处是中文摘要的格式设置,数据在cnabstract.tex %%%

\makeatletter

\long\def\cabstract#1{\long\def\@cabstract{#1}}\long\def\@cabstract{}
\def\ckeywords#1{\def\@ckeywords{#1}}\def\@ckeywords{}
\def\cnpageheader#1{\def\@cnpageheader{#1}}\def\@cnpageheader{}

% 定义makecnabstract以生成中文摘要
\def\makecnabstract{
	
	\clearpage
	\setcounter{page}{1} 		% 让摘要的起始页码为1。
	\pagestyle{plain}		
	%%%%%%%%%%%%%%%%%%%		中文摘要		%%%%%%%%%%%%%%%%%%%
	\addcontentsline{toc}{chapter}{摘\quad 要}
	\chapter*{\centering\erhao\song\bfseries\ 摘\qquad 要}	% 摘要标题,宋体加粗二号
	\song\defaultfont
	\vspace{\baselineskip} 										% 空两行
	\@cabstract													% 摘要正文
	\vspace{\baselineskip}
	
	%\hangafter=1\hangindent=52.3pt\noindent   %如果取消该行注释,关键词换行时将会自动缩进
	%\noindent
	{\xiaosi\song\textbf{关键词:} \@ckeywords}	% modified by Jon, 关键词,宋体小四
	
	\clearpage
}
\makeatother

%%%%%%%%%%%%%%%%%%%%%%


%%% 此处是英文摘要的格式设置,数据在enabstract.tex %%%
\makeatletter

\long\def\enabstract#1{\long\def\@enabstract{#1}}\long\def\@enabstract{}
\def\enkeywords#1{\def\@enkeywords{#1}}\def\@enkeywords{}
\def\enpageheader#1{\def\@enpageheader{#1}}\def\@enpageheader{}

% 定义makeenabstract以生成英文摘要
\def\makeenabstract{
	
	\clearpage
	
	% 使用fancy来设置页眉页脚,自定义了新的格式——englishAbstract
	\fancypagestyle{englishAbstract}{
		\fancyhead[C]{\wuhao \@enpageheader}		% 使用英文论文名称作为页眉
		\fancyfoot[C]{\song\wuhao ~\thepage~}
	}

	\pagestyle{englishAbstract}{}					% 应用englishAbstract
	
	\addcontentsline{toc}{chapter}{Abstract}
	\chapter*{\centering\erhao \bf{Abstract}}		% 标题,二号
	\thispagestyle{englishAbstract}{}				% 应用englishAbstract
	\vspace{\baselineskip} 							% 空两行
	
	\@enabstract									% 英文摘要正文
	\vspace{\baselineskip}
	
	%\hangafter=1\hangindent=60pt\noindent  %如果取消该行注释,KEY WORDS换行时将会自动缩进
	%\noindent
	{\xiaosi\textbf{Key Words:} \@enkeywords}		% 英文关键字
	
	\clearpage
}
\makeatother

%%%%%%%%%%%%%%%%%%%%%%